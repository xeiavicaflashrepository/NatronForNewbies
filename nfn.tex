\documentclass[12pt, oneside, openany, showtrims, letterpaper]{memoir}

\title{Natron For Newbies}
\author{Created By: Matthew Polk}
\date{}

% This will rename the title of TOC from "Contents" to something else.
\renewcommand*{\contentsname}{Table Of Contents}

% This will ensure that Table Of Contents page has no page number.
\AtBeginDocument{\addtocontents{toc}{\protect\thispagestyle{empty}}} 

%%<=====================================================================>
\begin{document}
\aliaspagestyle{title}{empty}
\maketitle

\newpage
\tableofcontents*
\pagestyle{empty}


\cleardoublepage

%%<====================================================================>
%
% TODO: Incorporate concepts from the following books used as reference:
% Nuke 101: Professional Compositor And Visual Effects (A lot of the principles can be applied to natron).
% The Visual Effects Producer
% The VES Handbook of Visual Effects
% The Filmmaker's Guide To Visual Effects
% The Art And Science Of Digital Compositing
% Professional Digital Compositing
% Digital Compositng for Film and Video
\setcounter{page}{1}
\chapter{Introduction}
% Purpose of this chapter is for introducing the reader to natron.

\chapter{Getting Around Natron}
% Purpose of this chapter is for how to interface with the GUI.

\chapter{Nodes}
% What are nodes? And how does one use them?

\chapter{What Is Compositing?}
% Purpose of this chapter is to explain intro to compositing fundamentals.

\chapter{Python API}
% Purpose of this chapter is to explain the python API

\chapter{GLSL}
% Purpose of this chapter is to introduce the reader to GLSL driven effects.




\end{document}
